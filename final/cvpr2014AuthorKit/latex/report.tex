\documentclass[10pt,twocolumn,letterpaper]{article}

\usepackage{cvpr}
\usepackage{times}
\usepackage{epsfig}
\usepackage{graphicx}
\usepackage{amsmath}
\usepackage{amssymb}

% Include other packages here, before hyperref.

% If you comment hyperref and then uncomment it, you should delete
% egpaper.aux before re-running latex.  (Or just hit 'q' on the first latex
% run, let it finish, and you should be clear).
\usepackage[breaklinks=true,bookmarks=false]{hyperref}

\cvprfinalcopy % *** Uncomment this line for the final submission

\def\cvprPaperID{****} % *** Enter the CVPR Paper ID here
\def\httilde{\mbox{\tt\raisebox{-.5ex}{\symbol{126}}}}

% Pages are numbered in submission mode, and unnumbered in camera-ready
%\ifcvprfinal\pagestyle{empty}\fi
% \setcounter{page}{4321}
\begin{document}

%%%%%%%%% TITLE
\title{CSE473 Final Project: Vision-based Micromouse Maze Wall Detection}

\author{Scott Will, Mack Ward\\
University at Buffalo\\
Amherst, NY 14261\\
{\tt\small scottwil@buffalo.edu, mward4@buffalo.edu}
% For a paper whose authors are all at the same institution,
% omit the following lines up until the closing ``}''.
% Additional authors and addresses can be added with ``\and'',
% just like the second author.
% To save space, use either the email address or home page, not both
% \and
% Mack Ward\\
% University at Buffalo\\
% {\tt\small mward4@buffalo.edu}
}

\maketitle
%\thispagestyle{empty}

%%%%%%%%% ABSTRACT
\begin{abstract}
	Since the late 1970s, students and professional engineers alike have competed in Micromouse, an event in which teams
	of participants construct wheeled robots that attempt to autonomously traverse and solve planar mazes in as little
	time as possible.  A common solution strategies adopted by competitors is a two-stage process: first, a mouse
	explores the maze and incrementally generates a virtual representation of the maze layout in its internal memory,
	then computes the optimal path from the starting position to the center of the maze, and finally returns to the
	starting position and moves along the optimal path.  Although this approach is effective and reliable, most of the
	total time spent in transit by the mouse tends to be devoted to the exploration phase.  To reduce this time, we
	propose a much different strategy: using a camera mounted on the mouse at sufficient height and techniques from
	computer vision and image processing, generate an image of the maze from the perspective of a viewer looking down
	from above, extract the locations of the maze walls programmatically, and proceed to solve the maze by directly
	moving to its center.
\end{abstract}

%%%%%%%%% BODY TEXT
\section{Introduction}
\label{sec:introduction}
%-------------------------------------------------------------------------

\subsection{Competition rules}
\label{subsec:rules}
In order to lessen the burden of constructing appropriately-sized vehicles, typical Micromouse events follow standard
rules and maze-guidelines.  We reproduce them here for thoroughness (adapted from the official rules from the IEEE
Region 1 Micromouse
Competition) []:

\subsubsection{Rules for the micromouse}
\label{subsubsec:mouserules}
\begin{enumerate}
	\item A Micromouse shall be self-contained (no remote controls). A Micromouse shall not use an energy source
	employing a combustion process.
	\item A Micromouse shall not leave any part of its body behind while negotiating the maze.
	\item A Micromouse shall not jump over, fly over, climb, scratch, cut, burn, mark, damage, or destroy
	the walls of the maze.
	\item A Micromouse shall not be larger either in length or in width, than 25 centimeters. The dimensions of a
	Micromouse that changes its geometry during a run shall not be greater than $25\ \textrm{cm} \times 25\
	\textrm{cm}$. There are no restrictions on the height of a Micromouse.
\end{enumerate}

\subsubsection{Rules for the maze}
\label{subsubsec:mazerules}
\begin{enumerate}
	\item The maze is composed of multiples of an $18\ \textrm{cm} \times 18\ \textrm{cm}$ unit square. The maze
       comprises $16 \times 16$ unit squares. The walls of the maze are $5\ \textrm{cm}$ high and $1.2\ \textrm{cm}$
       thick (assume 5\% tolerance for mazes). The outside wall encloses the entire maze.
	\item The sides of the maze walls are white, the tops of the walls are red, and the floor is black. The maze is made
       of wood, finished with non-gloss paint.
	\item The start of the maze is located at one of the four corners. The start square is bounded on three sides by
       walls. The start line is located between the first and second squares. That is, as the mouse exits the corner
       square, the time starts. The destination goal is the four cells at the center of the maze. At the center of this
       zone is a post, $20\ \textrm{cm}$ high and each side $2.5\ \textrm{cm}$. (This post may be removed if requested.)
       The destination square has only one entrance.
	\item Small square zones (posts), each $1.2\ \textrm{cm} \times 1.2\ \textrm{cm}$, at the four corners of each unit
       square are called lattice points. The maze is so constituted that there is at least one wall at each lattice
       point.
	\item Multiple paths to the destination square are allowed and are to be expected. The destination square will be
       positioned so that a wall-hugging mouse will NOT be able to find it.
\end{enumerate}

\subsection{Common Solution Methods}
\label{subsec:solutionmethods}
As briefly described above, a great majority of teams tend to follow roughly the same high-level steps for producing a
mouse capable of reaching the center of a maze meeting the above specifications:

\begin{enumerate}
	\item Fully traverse the maze (including all dead-end routes) and generate an internal representation of the maze
	topology
	\item Use one or more searching algorithms such as bread-first or depth-first search, to find the optimal path from the starting point to the maze center
	\item Move along the optimal path to reach the maze center
\end{enumerate}

A less-popular method is the so-called "wall-follower" approach, which seeks to eventually reach the center of the maze
by simply moving along a given wall for its entire length.  This is not guaranteed to be effective for all possible maze
structures, and is officially discouraged by the competition organizers (see \textsection\ref{subsubsec:mazerules}).

\subsection{Our Solution}
\label{subsec:oursolution}
As members of the IEEE student chapter at UB, we have competed in the annual IEEE Region 1 Micromouse Competition, which
is open to student teams from universities across the northeastern United States, twice.  We conceived the idea to
engineer a vision-based Micromouse implementation after observing the weaknesses of the "traditional" style of
implementation, which, in our opinion, spent too much unnecessary time determining maze layouts before being able to
arrive at, and traverse, solutions.  We realized that because of the regularity of competition mazes, which were
composed of equally-sized unit squares, uniform-height walls, and so on, that it might be possible to extract all of the
necessary information to produce an accurate representation of maze layout using only visual data obtainable from the
Micromouse's perspective.

\section{Related Work}
\label{sec:relatedwork}
We are not the first to suggest using tools from computer vision to improve the capabilities of Micromouse
implementations.  In 1997, Ning Chen suggested to revise the Micromouse competition to encourage solutions more relevant to real-world engineering challenges, by introducing uneven mazes resembling real 


\section{Algorithm}
\label{sec:algorithm}
\section{Results}
\label{sec:results}
\section{Discussion}
\label{sec:discussion}
{\small
\bibliographystyle{ieee}
\bibliography{references}
}

\end{document}
